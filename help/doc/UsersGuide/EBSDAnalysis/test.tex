
\documentclass{article}
\usepackage{graphicx}
\usepackage{color}

\sloppy
\definecolor{lightgray}{gray}{0.5}
\setlength{\parindent}{0pt}

\begin{document}



		\chapter{User's Guide}

		
               \begin{par}
Provide tutorials and comprehensive product information for the MTEX Toolbox
\end{par} \vspace{1em}

            
		\section{Crystal Geometry}

		
                  \begin{par}
Introduces key concepts about the MTEX representation of specimen directions, crystal directions, crystal symmetries, rotations and orientations.
\end{par} \vspace{1em}

               
		\subsection{Specimen Directions}

		
                     \begin{par}
Explains how to define and calculate with specimen directions.
\end{par} \vspace{1em}

                  
			\paragraph{Open in Editor}
		
			\paragraph{Defining Specimen Directions}
		
			\paragraph{Calculating with Specimen Directions}
		
			\paragraph{Plotting three dimensionl vectors}
		
		\subsection{Crystal Symmetries}

		
                     \begin{par}
Explains how to define crystal symmetries.
\end{par} \vspace{1em}

                  
			\paragraph{Open in Editor}
		
			\paragraph{Defining a Crystal Symmetry by the Name of its Laue Group}
		
			\paragraph{Defining a Crystal Symmetry by the Name of its Point Group or its Space Group}
		
			\paragraph{Defining a Crystal Symmetry by a CIF File}
		
			\paragraph{The Crystal Coordinate System}
		
			\paragraph{A and B Configurations}
		
			\paragraph{Calculations}
		
			\paragraph{Plotting symmetries}
		
		\subsection{Rotations}

		
                     \begin{par}
Explains how to define rotations and how to switch between different Euler angle conventions.
\end{par} \vspace{1em}

                  
			\paragraph{Open in Editor}
		
			\paragraph{Description}
		
			\paragraph{Euler Angle Conventions}
		
			\paragraph{Other Ways of Defining a Rotation}
		
			\paragraph{Calculating with Rotations}
		
			\paragraph{Conversion into Euler Angles and Rodrigues Parametrisation}
		
			\paragraph{Plotting Rotations}
		
		\subsection{Crystal Orientations}

		
                     \begin{par}
Explains how to define crystal orientations, how to switch between different convention and how to compute crystallographic equivalent orientations.
\end{par} \vspace{1em}

                  
			\paragraph{Open in Editor}
		
			\paragraph{Defining a Crystal Orientation}
		
			\paragraph{Rotating Crystal Directions onto Specimen Directions}
		
			\paragraph{Concatenating Rotations}
		
			\paragraph{Caclulating Missorientations}
		
			\paragraph{Calculating with Orientations and Rotations}
		
			\paragraph{Conversion into Euler Angles and Rodrigues Parametrisation}
		
			\paragraph{Plotting Orientations}
		
		\subsection{Crystal Directions}

		
                     \begin{par}
Explains how to define crystal directions by Miller indices and how to compute crystallographic equivalent directions.
\end{par} \vspace{1em}

                  
			\paragraph{Open in Editor}
		
			\paragraph{Definition}
		
			\paragraph{Plotting Miller indece}
		
			\paragraph{Symmetrically Equivalent Crystal Directions}
		
			\paragraph{Angles}
		
			\paragraph{Conversations}
		
			\paragraph{Calculations}
		
		\subsection{Antipodal Symmetry}

		
                     \begin{par}
Explains the MTEX option antipodal and the impact of antipodal symmetry to pole figure plots and EBSD colorcoding.
\end{par} \vspace{1em}

                  
			\paragraph{Open in Editor}
		
			\paragraph{Directions vs. Axes}
		
			\paragraph{The Angle between Directions and Axes}
		
			\paragraph{Antipodal Symmetry in Experimental Pole Figures}
		
			\paragraph{Antipodal Symmetry in Recalculated Pole Figures}
		
			\paragraph{Antipodal Symmetry in inverse Pole Figures}
		
			\paragraph{EBSD Colocoding}
		
		\section{Pole Figure Analysis}

		
                  \begin{par}
Explains how to import pole figure data, how to correct them, and how to recover an ODF.
\end{par} \vspace{1em}

               
		\subsection{Modify Pole Figure Data}

		
                     \begin{par}
Explains how to manipulate pole figure data in MTEX.
\end{par} \vspace{1em}

                  
			\paragraph{Open in Editor}
		
			\paragraph{Import diffraction data}
		
			\paragraph{Splitting and Reordering of Pole Figures}
		
			\paragraph{Correct pole figure data}
		
			\paragraph{Normalize pole figures}
		
			\paragraph{Modify certain pole figure values}
		
			\paragraph{Remove certain measurements from the data}
		
			\paragraph{Rotate pole figures}
		
		\subsection{Plotting of Pole Figures}

		
                     \begin{par}
Described various possibilities to visualize pole figure data.
\end{par} \vspace{1em}

                  
			\paragraph{Open in Editor}
		
			\paragraph{Import of Pole Figures}
		
			\paragraph{Visualize the Data}
		
			\paragraph{Contour Plots}
		
			\paragraph{Plotting Recalculated Pole Figures}
		
		\subsection{Ghost Effect Analysis}

		
                     \begin{par}
Explains the ghost effect to ODF reconstruction and the MTEX option ghostcorrection.
\end{par} \vspace{1em}

                  
			\paragraph{Open in Editor}
		
			\paragraph{Introduction}
		
			\paragraph{Construct Model ODF}
		
			\paragraph{Simulate pole figures}
		
			\paragraph{ODF Estimation}
		
			\paragraph{Compare RP Errors}
		
			\paragraph{Compare Reconstruction Errors}
		
			\paragraph{Plot the ODFs}
		
			\paragraph{Calculate Fourier coefficients}
		
			\paragraph{Calculate Reconstruction Errors from Fourier Coefficients}
		
			\paragraph{Plot Fourier Coefficients}
		
		\subsection{ODF Estimation from Pole Figure Data}

		
                     \begin{par}
This page describes how to use MTEX to estimate an ODF from pole figure data.
\end{par} \vspace{1em}

                  
			\paragraph{Open in Editor}
		
			\paragraph{ODF Estimation}
		
			\paragraph{Error analyis}
		
			\paragraph{Discretization}
		
			\paragraph{Zero Range Method}
		
			\paragraph{Ghost Corrections}
		
			\paragraph{Theory}
		
		\subsection{Importing Pole Figure Data}

		
                     \begin{par}
How to import Pole Figure Data
\end{par} \vspace{1em}

                  
			\paragraph{Import pole figure data using the import wizard}
		
			\paragraph{List of automatically detected interfaces}
		
			\paragraph{Importing pole figure data using the method loadPoleFigure}
		
			\paragraph{Importing pole figure data from general ascii files}
		
			\paragraph{Importing pole figure data from unknown formats}
		
			\paragraph{Writing your own interface}
		
		\subsection{Pole Figure Simulation}

		
                     \begin{par}
Simulate arbitary pole figure data
\end{par} \vspace{1em}

                  
			\paragraph{Open in Editor}
		
			\paragraph{Introduction}
		
			\paragraph{Simulate Pole Figure Data}
		
			\paragraph{ODF Estimation from Pole Figure Data}
		
			\paragraph{Exploration of the relationship between estimation error and number of single orientations}
		
		\subsection{First Steps}

		
                     \begin{par}
Get in touch with PoleFigure Data in MTEX.
\end{par} \vspace{1em}

                  
			\paragraph{Open in Editor}
		
			\paragraph{Import of Pole Figures}
		
			\paragraph{Plotting Pole Figure Data}
		
			\paragraph{Modify Pole Figure Data}
		
			\paragraph{Calculate an ODF from Pole Figure Data}
		
			\paragraph{Simulate Pole Figure Data}
		
		\section{EBSD Data Analysis}

		
                  \begin{par}
Data Import of Electron Backscatter Diffraction Data, Correct Data, Estimeate Orientation Density Functions out of EBSD Data, Model Grains and Misorientation Density Functions
\end{par} \vspace{1em}

               
		\subsection{Import EBSD Data}

		
                     \begin{par}
How to import EBSD Data
\end{par} \vspace{1em}

                  
			\paragraph{Importing EBSD data using the import wizard}
		
			\paragraph{Features of the import wizard}
		
			\paragraph{Supported Interfaces}
		
			\paragraph{The generic cnterface}
		
			\paragraph{The Import Script}
		
			\paragraph{Writing your own interface}
		
			\paragraph{See also}
		
		\subsection{Modify EBSD Data}

		
                     \begin{par}
How to use MTEX to correct EBSD data for measurement errors.
\end{par} \vspace{1em}

                  
			\paragraph{Open in Editor}
		
			\paragraph{Realign / Rotate the data}
		
			\paragraph{See also}
		
			\paragraph{Restricting to a region of interest}
		
			\paragraph{Remove Inaccurate Orientation Measurements}
		
		\subsection{Grain Statistics}

		
                     \begin{par}
Access properties of grains to perfom statistics.
\end{par} \vspace{1em}

                  
			\paragraph{Open in Editor}
		
			\paragraph{Grain-size Analysis}
		
			\paragraph{Spatial Dependencies}
		
		\subsection{Misorientation Analysis}

		
                     \begin{par}
Analyse interior misorientation and misorientation between neighboured grains.
\end{par} \vspace{1em}

                  
			\paragraph{Open in Editor}
		
			\paragraph{Intergranular Misorientation}
		
			\paragraph{Boundary Misorientation}
		
		\subsection{ODF Estimation from EBSD data}

		
                     \begin{par}
How to estimate an ODF from single orientation measurements.
\end{par} \vspace{1em}

                  
			\paragraph{Open in Editor}
		
			\paragraph{ODF Estimation}
		
			\paragraph{Automatic halfwidth selection}
		
			\paragraph{Effect of halfwidth selection}
		
		\subsection{Grain Reconstruction}

		
                     \begin{par}
Explanation how to create grains from EBSD data.
\end{par} \vspace{1em}

                  
			\paragraph{Open in Editor}
		
			\paragraph{Grain Reconstruction}
		
			\paragraph{Grain properties}
		
			\paragraph{Connection between EBSD Data and a Grains}
		
		\subsection{Plotting Individual Orientations}

		
                     \begin{par}
Basics to the plot types for individual orientations data
\end{par} \vspace{1em}

                  
			\paragraph{Open in Editor}
		
			\paragraph{Scatter Pole Figure Plot}
		
			\paragraph{Scatter (Inverse) Pole Figure Plot}
		
			\paragraph{Scatter Plot in ODF Sections}
		
			\paragraph{Scatter Plot in Axis Angle or Rodrigues Space}
		
			\paragraph{Orientation plots for EBSD and grains}
		
		\subsection{Plotting Spatial Orientation Data}

		
                     \begin{par}
Howto plot spatially indexed orientations
\end{par} \vspace{1em}

                  
			\paragraph{Open in Editor}
		
			\paragraph{Coloring spatially orientation data}
		
			\paragraph{Visualising Grain Boundaries}
		
			\paragraph{Coloring other properties}
		
		\section{ODF Analysis}

		
                  \begin{par}
Explains how to import and export ODFs, how to define model ODFs and how to analyze ODFs, e.g., with respect to modalorientations, textureindex, volumeportions. Pole figure simulation and single orientation simulation is explained as well.
\end{par} \vspace{1em}

               
		\subsection{ODF Characteristics}

		
                     \begin{par}
Explains how to analyze ODFs, i.e. how to compute modal orientations, texture index, volume portions, Fourier coefficients and pole figures.
\end{par} \vspace{1em}

                  
			\paragraph{Open in Editor}
		
			\paragraph{Modal Orientations}
		
			\paragraph{Texture Characteristics}
		
			\paragraph{Volume Portions}
		
			\paragraph{Fourier Coefficients}
		
			\paragraph{Pole Figures and Values at Specific Orientations}
		
			\paragraph{Extract Internal Representation}
		
		\subsection{Visualising ODFs}

		
                     \begin{par}
Explains all possibilities to visualize ODfs, i.e. pole figure plots, inverse pole figure plots, ODF sections, fibre sections.
\end{par} \vspace{1em}

                  
			\paragraph{Open in Editor}
		
			\paragraph{Pole Figures}
		
			\paragraph{Inverse Pole Figures}
		
			\paragraph{ODF Sections}
		
			\paragraph{3D Euler Space}
		
			\paragraph{One Dimensional ODF Sections and Fibres}
		
			\paragraph{Fourier Coefficients}
		
			\paragraph{Axis / Angle Distribution}
		
		\subsection{EBSD Simulation}

		
                     \begin{par}
How to simulate an arbitary number of individual orientations data from any ODF.
\end{par} \vspace{1em}

                  
			\paragraph{Open in Editor}
		
			\paragraph{Simulate EBSD Data}
		
			\paragraph{ODF Estimation from EBSD Data}
		
			\paragraph{Exploration of the relationship between estimation error and number of single orientations}
		
		\subsection{ODF Estimation from Pole Figure Data}

		
                     \begin{par}
This page describes how to use MTEX to estimate an ODF from pole figure data.
\end{par} \vspace{1em}

                  
			\paragraph{Open in Editor}
		
			\paragraph{ODF Estimation}
		
			\paragraph{Error analyis}
		
			\paragraph{Discretization}
		
			\paragraph{Zero Range Method}
		
			\paragraph{Ghost Corrections}
		
			\paragraph{Theory}
		
		\subsection{ODF Estimation from EBSD data}

		
                     \begin{par}
How to estimate an ODF from single orientation measurements.
\end{par} \vspace{1em}

                  
			\paragraph{Open in Editor}
		
			\paragraph{ODF Estimation}
		
			\paragraph{Automatic halfwidth selection}
		
			\paragraph{Effect of halfwidth selection}
		
		\subsection{Pole Figure Simulation}

		
                     \begin{par}
Simulate arbitary pole figure data
\end{par} \vspace{1em}

                  
			\paragraph{Open in Editor}
		
			\paragraph{Introduction}
		
			\paragraph{Simulate Pole Figure Data}
		
			\paragraph{ODF Estimation from Pole Figure Data}
		
			\paragraph{Exploration of the relationship between estimation error and number of single orientations}
		
		\subsection{Model ODFs}

		
                     \begin{par}
Describes how to define model ODFs in MTEX, i.e., uniform ODFs, unimodal ODFs, fibre ODFs, Bingham ODFs and ODFs defined by its Fourier coefficients.
\end{par} \vspace{1em}

                  
			\paragraph{Open in Editor}
		
			\paragraph{Introduction}
		
			\paragraph{The Uniform ODF}
		
			\paragraph{Unimodal ODFs}
		
			\paragraph{Fibre ODFs}
		
			\paragraph{ODFs given by Fourier coefficients}
		
			\paragraph{Bingham ODFs}
		
			\paragraph{Combining MODEL ODFs}
		
		\subsection{Import and Export of ODF Data}

		
                     \begin{par}
Read and write ODFs to a Data file
\end{par} \vspace{1em}

                  
			\paragraph{Open in Editor}
		
			\paragraph{Define an Model ODF}
		
			\paragraph{Export an ODF to an MTEX ASCII File}
		
			\paragraph{Export as an generic ASCII file}
		
			\paragraph{Import ODF Data using the import wizard}
		
			\paragraph{Importing EBSD data using the method loadODF}
		
		\section{Material Tensors}

		
                  \begin{par}
Explains how to work with material tensors in MTEX, i.e. how to compute mean material tensors according to an ODF or to EBSD data, how to compute rotate and visualize tensors and how to calculate with elasticity tensors.
\end{par} \vspace{1em}

               
		\subsection{Tensor Arithmetics}

		
                     \begin{par}
how to calculate with tensors in MTEX
\end{par} \vspace{1em}

                  
			\paragraph{Open in Editor}
		
			\paragraph{Abstract}
		
			\paragraph{Defining a Tensor}
		
			\paragraph{Importing a Tensor from a File}
		
			\paragraph{Visualization}
		
			\paragraph{Rotating a Tensor}
		
			\paragraph{The Inverse Tensor}
		
			\paragraph{Tensor Products}
		
		\subsection{Average Material Tensors}

		
                     \begin{par}
how to calculate average material tensors from ODF and EBSD data
\end{par} \vspace{1em}

                  
			\paragraph{Open in Editor}
		
			\paragraph{Abstract}
		
			\paragraph{Import EBSD Data}
		
			\paragraph{Data Correction}
		
			\paragraph{Define Elastic Stiffness Tensors for Glaucophane and Epidote}
		
			\paragraph{The Average Tensor from EBSD Data}
		
			\paragraph{ODF Estimation}
		
			\paragraph{The Average Tensor from an ODF}
		
		\subsection{The Elasticity Tensor}

		
                     \begin{par}
how to calculate the elasticity properties
\end{par} \vspace{1em}

                  
			\paragraph{Open in Editor}
		
			\paragraph{Abstract}
		
			\paragraph{Import an Elasticity Tensor}
		
			\paragraph{Young's Modulus}
		
			\paragraph{Linear Compressibility}
		
			\paragraph{Cristoffel Tensor}
		
			\paragraph{Elastic Wave Velocity}
		
		\section{Plotting}

		
                  \begin{par}
Explains different plot types and how to customize them, inlcuding annotations, spherical projections, color coding.
\end{par} \vspace{1em}

               
		\subsection{Spherical Projections}

		
                     \begin{par}
Explains the spherical projections MTEX offers for plotting crystal and specimen directions, pole figures and ODF.
\end{par} \vspace{1em}

                  
			\paragraph{Open in Editor}
		
			\paragraph{Introduction}
		
			\paragraph{Alignment of the Hemishpheres}
		
			\paragraph{Alignment of the Coordinate Axes}
		
			\paragraph{Equal Area Projection (Schmidt Projection)}
		
			\paragraph{Equal Distance Projection}
		
			\paragraph{Stereographic Projection (Equal Angle Projection)}
		
			\paragraph{Plain Projection}
		
			\paragraph{Three Dimensional Plots}
		
		\subsection{Plotting Overview}

		
                     \begin{par}
Overview over the plotting faccilities of MTEX, including annotations, plot types, color coding, combined plots and export of plots.
\end{par} \vspace{1em}

                  
			\paragraph{Open in Editor}
		
			\paragraph{Plotting in MTEX}
		
			\paragraph{Plot Types}
		
			\paragraph{Spherical Projections}
		
			\paragraph{Color Coding}
		
			\paragraph{Plot Annotations}
		
			\paragraph{Combined Plots}
		
			\paragraph{Exporting Plots}
		
			\paragraph{Changing the Default Plotting Options}
		
		\subsection{Pole Figure Color Coding}

		
                     \begin{par}
Explains how to control color coding across multiple plots.
\end{par} \vspace{1em}

                  
			\paragraph{Open in Editor}
		
			\paragraph{Abstract}
		
			\paragraph{A sample ODFs and Simulated Pole Figure Data}
		
			\paragraph{Setting a Colormap}
		
			\paragraph{Tight Colorcoding}
		
			\paragraph{Equal Colorcoding}
		
			\paragraph{Setting an Explicite Colorrange}
		
			\paragraph{Setting the Contour Levels}
		
			\paragraph{Modifying the Colorrange After Plotting}
		
			\paragraph{Logarithmic Plots}
		
		\subsection{Annotations}

		
                     \begin{par}
Explains how to add annotations to plots. This includes colorbars, legends, specimen directions and crystal directions.
\end{par} \vspace{1em}

                  
			\paragraph{Open in Editor}
		
			\paragraph{Some sample ODFs}
		
			\paragraph{Adding a Colorbar}
		
			\paragraph{Adding Specimen and Crystal Directions}
		
			\paragraph{Adding Preferred Orientations}
		
			\paragraph{Adding a Legend}
		
			\paragraph{Adding a Spherical Grid}
		
		\subsection{Combinded Plots}

		
                     \begin{par}
Explains how to combine several plots, e.g. plotting on the top of an inverse pole figure some important crystall directions.
\end{par} \vspace{1em}

                  
			\paragraph{Open in Editor}
		
			\paragraph{General Principle}
		
			\paragraph{Combine Different EBSD Data}
		
			\paragraph{Combine Smooth ODF Plots with EBSD Data Scatter Plots}
		
			\paragraph{Add Miller Indece to a Inverse Pole Figure Plot}
		
		\subsection{EBSD Color Coding}

		
                     \begin{par}
Explains how to control EBSD color coding.
\end{par} \vspace{1em}

                  
			\paragraph{Open in Editor}
		
			\paragraph{See also}
		
			\paragraph{IPDF Colorcoding}
		
			\paragraph{IPDF Overview}
		
			\paragraph{Other Colorcodes}
		
		\subsection{Plot Types}

		
                     \begin{par}
Explains the different plot types, i.e., scatter plots, contour plots, and line plots.
\end{par} \vspace{1em}

                  
			\paragraph{Open in Editor}
		
			\paragraph{A Sample ODFs}
		
			\paragraph{Scatter Plots}
		
			\paragraph{Contour Plots}
		
			\paragraph{Filled Contour Plots}
		
			\paragraph{Smooth Interpolated Plots}
		
			\paragraph{Line Plots}
		
		\section{Import and Export of Data}

		
                  \begin{par}
How to import and to export Pole Figure Data, EBSD Data and ODFs
\end{par} \vspace{1em}

               
			\paragraph{Introduction}
		
		\subsection{Import and Export of ODF Data}

		
                     \begin{par}
Read and write ODFs to a Data file
\end{par} \vspace{1em}

                  
			\paragraph{Open in Editor}
		
			\paragraph{Define an Model ODF}
		
			\paragraph{Export an ODF to an MTEX ASCII File}
		
			\paragraph{Export as an generic ASCII file}
		
			\paragraph{Import ODF Data using the import wizard}
		
			\paragraph{Importing EBSD data using the method loadODF}
		
		\subsection{Importing Pole Figure Data}

		
                     \begin{par}
How to import Pole Figure Data
\end{par} \vspace{1em}

                  
			\paragraph{Import pole figure data using the import wizard}
		
			\paragraph{List of automatically detected interfaces}
		
			\paragraph{Importing pole figure data using the method loadPoleFigure}
		
			\paragraph{Importing pole figure data from general ascii files}
		
			\paragraph{Importing pole figure data from unknown formats}
		
			\paragraph{Writing your own interface}
		
		\subsection{Import EBSD Data}

		
                     \begin{par}
How to import EBSD Data
\end{par} \vspace{1em}

                  
			\paragraph{Importing EBSD data using the import wizard}
		
			\paragraph{Features of the import wizard}
		
			\paragraph{Supported Interfaces}
		
			\paragraph{The generic cnterface}
		
			\paragraph{The Import Script}
		
			\paragraph{Writing your own interface}
		
			\paragraph{See also}
		

 
\end{document}
